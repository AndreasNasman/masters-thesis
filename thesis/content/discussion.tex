\bgroup{}

\chapter{Discussion}\label{cha:discussion}
The developed application fulfills the goal of enabling subjective tagging of sounds while minimizing the manual effort needed to do so, hence proving that the proposed system is a viable and working solution. Since there exist no prior comparable solutions, at least not found by the author, it is impossible to measure the performance of the program to any previous values.

The end product is a standalone application capable of tagging sounds according to different opinions and continuously improves with usage. The solution is in itself quite simple and relatively easy to adapt. When narrowing down the codebase to units responsible for actual calculations related to sound similarity, tagging, and grouping, the result is only a couple of hundred lines of code. Still, the intended purpose of the solution is to function as a complement to other existing systems. This expansion could be in the form of a plugin, an extension, or some other similar variation.

Many of the components in the implemented program are competent and well-functioning. The sound similarity system and the automatic tagging functionalities perform well in all of the use cases evaluated in \cref{cha:evaluation}. Regardless of how specific or generic the desired tag distribution might be, it should always be possible to achieve the sought after result using the application as a platform in its current form.

As described in the evaluation process, there are many ways of reaching a finalized state where all sounds are correctly labeled, depending on what order sounds are tagged. Each available path requires a specific amount of corrections, meaning that some paths need more manual effort to complete than others. Therefore, the course taken by the user might end up being the most effortless one, the most tedious one, or something in between.

The application is deterministic in the sense that repeating the same actions always result in the same outcome, which makes it stable and possibly easier to integrate with other systems. However, the program is inconsistent with tag distributions and groups, as the corresponding sum of groups and how they encompass the sounds can vary in a fixed arrangement of labels. Tagging multiple sounds in one way might conclude in a few groups, while in another, the number of groups could be doubling.

While developing and formulating the proposed solution, the author of this thesis realized that many different kinds of elements play a role when forming a subjective categorization and management system for sounds. The fundamental dilemma concerning subjectivity is the most difficult to understand and define fully. Because subjectivity is abstract and can change from situation to situation, it is difficult to make assumptions around it and define rules. Subjectivity is also presumptively exposed to peer pressure; if someone describes a sound as, e.g., `sad', someone else will probably perceive the sound the same way. If the latter person heard the sound in isolation, they might have labeled the sound differently, unaffected by the other person's predefined description. These kinds of human factors can make it challenging to pinpoint problems when measuring the effectiveness of categorization and management systems for sounds, as the user could have trouble understanding or describing the problem themselves. When annotating sounds, adapting one's perception to someone else's might work for some cases, but resisting this tendency and creating a personal collection of labels instead will hypothetically be beneficial in the long run.

\egroup{}