\bgroup{}

\chapter{Conclusion}\label{cha:conclusion}
Designing sounds and building sound landscapes is a challenging yet inspiring occupation. Dealing with large data sets and having numerous influencing factors are some of the concerns making the situation more demanding. Utilizing sound libraries containing \glspl{sfx} is one way to make the process more approachable and productive. However, many aspects still need improvement to make the means of working with the libraries more gratifying and compelling. No standard exists for categorizing and classifying sounds, meaning that most libraries have a custom labeling system, making it harder to develop programs capable of handling all variations. The tools used for finding and managing sounds among the libraries are also far from perfect.

Extensive research has taken place in automatic sound identification and objective audio labeling, which could help classify sounds more accurately and expressly. The research results and developed systems thereof could assist the library creators in the tagging process, making related sounds across multiple libraries labeled similarly. This procedure could aid in establishing a standard of how to describe sounds concretely, such as the sound source of the audio.

Far less experimentation has ensued with systems aimed towards individual sound annotation with personal descriptors, capable of adjusting to different users' perception. Humans often describe signals perceived through the senses with subjective words, e.g., that something is scary. This impression holds for audio as well: one person might interpret a sound as `happy' when another defines it as `smooth.'

The purpose of this thesis is to present and evaluate an approach capable of categorizing sounds with subjective tags. The developed solution is a flexible stand-application application, which continuously improves with usage and minimizes the manual effort needed to label sounds utilizing a sound similarity and automatic tagging system. Although the program functions well on its own, the eventual intent is to have the system integrated with existing categorization methods and management tools. This procedure could hopefully improve the work process for sound designers using sound libraries and also spark ideas for new programs or extensions to systems regarding the subject.

\section{Further work}\label{sec:further_work}
There is plenty of room for improvement and optimization concerning the application described in this thesis, as the program in its current stage only ranks as a prototype. Many of these advancements relate to the performance or usability aspects of the application, while others propose enhancements to the underlying logic.

One of the main issues that prevent the present version of the design from becoming a distributable application or integrated into existing systems is the lack of functionality for multiple tags per sound. Multi-labeling coheres substantially more with real-world use cases than only allowing individual tags, as a single label is seldom enough to describe a sound fully incorporating all its qualities. However, managing multiple tags require a different approach capable of handling advanced situations, such as overlapping and intertwining labels. Extending the described implementation to support this feature would probably lead to a complete rewrite, possibly preserving some of the current system design.

Another central subject needing further work is the sound analysis part of the application regarding both performance and accuracy. The sound similarity concept described in \cref{sec:sound_similarity} relies on its building blocks being accurate, which are the seven predicted timbral characteristics used throughout the program. The original purpose of the values is to assist in automatic labeling of sounds, and adapting them to serve as subjectivity measurement variables could lead to inconsistent and inadequate results. Also, the absence of dimensionality reduction and other forms of optimization procedures means some of the characteristics might be redundant.

Furthermore, when processing extensive data sets, the analysis step takes a considerable amount of time to finish. As of now, the application processes sounds one after the other, resulting in the operation taking longer for each additional sound added to the pool of sounds to analyze. This issue is a significant bottleneck performance problem preventing the system from scaling. Some form of parallelization, changing the single-threaded sequence of events, would cut down the execution time significantly. Massive data sets also slow down the program considerably, preventing any form of testing when the number of included sounds is high. This issue makes it difficult to check the performance scalability of the implemented \gls{ml} algorithms.

Both the \gls{ml} algorithms used for grouping sounds accomplish their shared goal of categorizing sounds satisfactory in their current form. Nevertheless, as touched upon in \cref{sec:case_3}, the algorithms could benefit from revision, since the working configurations incidentally use the default values defined in the package. One advantageous or disadvantageous outcome – depending on the situation – of using the predefined versions of the algorithms is how they directly adapt to changing the scale of how widespread added sounds are. Adding numerous closely resembling sounds could result in the same initial group predictions as a collection of more spread out sounds. Despite its situational usefulness, this behavior is currently persistent at all times. Having no control over this dynamical tolerance means that it activates for situations where it is a hindrance, as well as instances where it is beneficial. Tweaking and controlling the precision at which the algorithms draw their group borders might improve the performance in many cases. The degree at which this transpires and what the optimal tolerance comes out as is probably dependant on each user's sound perception.

How sounds are tagged and retagged might also be slightly confusing to the user. The tagging process currently depends on the sounds' verification status and decides the appropriate choices accordingly on its own in encapsulation. This implementation is a working solution in its own right but has its pros and cons. The program takes responsibility for most of the choices concerning tagging at the cost of the user having less control. Eliminating the verification property and implementing separate functionalities for creating and modifying tags could make the application more approachable. Some form of internal verification system would still need to be present to keep track of the confirmed tagged sounds. However, this means handling edge cases appropriately and redesigning the \gls{ui} accordingly. Regarding updating the visuals, this would naturally occur when implementing the proposed solution into an existing system. The design built with the \emph{tkinter} package works fine for presenting the prototype stage of the program but will need an overhaul to meet the industry standard.

\egroup{}