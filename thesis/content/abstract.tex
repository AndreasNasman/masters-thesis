\bgroup{}

\chapter*{\makebox[\textwidth]{Abstract}}
Sound designers use extensive sound libraries together with associated metadata management tools in their everyday life. However, there exists no standard telling how to organize or how to label the audio files. The maintainers of the libraries annotate the sounds with descriptive labels that specify the source of the audio, the materials used, or some other physical property about the samples. As this process is both an error-prone and a time-consuming task, research has taken place, producing automatic sound identification and categorization methods. An area far less covered is the consideration and application of subjectivity in these systems.

This thesis explores a theoretical and practical solution of subjective audio tagging in the form of a developed program, focusing on the individual. The proposed system demonstrates an application capable of finding similar groups of sound according to a specific user's perception and automatic tagging of these sounds. Continuous improvement applies to the system with the help of unsupervised machine learning run recursively. Publicly available audio packs simulate segregation of sound according to different persons, which help test and evaluate the proposed solution with various example use cases.

\bigskip
\bigskip

\noindent\textit{\textbf{Keywords}: audio identification, audio classification, machine learning, metadata management, sound design, sound effects, tagging ontology, tagging recommendation}

\egroup{}