\bgroup{}

\chapter{Introduction}
Sound plays a more vital role than ever in today's media-influenced world. \Glspl{sfx}, jingles, and other snippets of sound bring out a devised experience among listeners, either separately or together in forms of larger unities. Different applications may require distinct approaches and specializations to achieve the desired sound landscape, but the methods used for developing the needed sounds are mostly the same.

Sound design is the practice of bringing sounds to life intended for a specific production~\cite[1]{rep:knowledge_and_content-based_audio_retrieval_using_wordnet}. Many sound designers choose to utilize sound libraries with predefined samples to speed up the work process. Since the size of the sound libraries can be quite massive, various management tools are available to improve the accessibility of the libraries, making it easier to organize and find wanted sounds. However, both the libraries themselves and the accompanying tools have flaws and drawbacks in the way they are structured and how they function.

\section{Motive}\label{sec:motive}
The purpose of this thesis is to suggest an improvement to the current management tools for sound libraries. A shared shortcoming in all of the tools is the lack of adaptability to different users' understanding and judgment regarding sound perception. The proposed solution is a flexible and self-improving system empowering personal organization of sounds using subjective tags. To explain the approach more clearly, the author of the thesis has implemented a prototype application of the system that demonstrates a practical approach with minimal manual tagging effort needed from the user. The program establishes a frame of similar sounds according to a user's perception and eases the tagging process with automatic labeling. The solution forms around the hypothesis that the comprehension of sound differs enough among people that a system like this is feasible~\cite[2]{rep:psychoacoustically_informed_spectrography_and_timbre}. Existing categorization systems and management tools can hopefully adopt the program and embellish it, therefore improving the craft for sound designers making use of them.

\section{Thesis outline}
This thesis consists of three introductory chapters, including the current one. The present chapter provides a brief background on the chosen subject, defines the purpose of the thesis, and maps out the overall structure of the paper. \Cref{cha:background} extends on the background of the topic, describes relevant problems, and reviews related work. \Cref{cha:methods} provides detailed information on the methods applied to achieve the proposed solution.

\Cref{cha:implementation} and \cref{cha:evaluation} explain the implemented application and how it performs in different scenarios. The former illustrates the design of the program and demonstrates the underlying code, and the latter specifies the evaluation process and measures the system using four example use cases.

Lastly, two chapters wrap up the thesis in the form of a discussion about the project in \cref{cha:discussion} and a concluding summary, plus analysis on future work, in \cref{cha:conclusion}.

\egroup{}